\documentclass[11pt]{article} % use larger type; default would be 10pt

\usepackage[utf8]{inputenc} % set input encoding (not needed with XeLaTeX)
\usepackage{bm}

\usepackage{geometry} % to change the page dimensions
\geometry{letterpaper} % or letterpaper (US) or a5paper or....
%%% The "real" document content comes below...

\title{Quantitative Analysis of the Benefits of User-Aware Mass Transit Networks}
\author{The Author}
\date{}

\begin{document}
\maketitle

\section*{Problem 1: }
There is a bus that will traverse a line segment $S$ from $\bm{\vec{b}_1} = (x_1, y_1)$ to  $\bm{\vec{b}_2} = (x_2, y_2)$ at speed $v_b$, starting at some time $t_1$ potentially in the future ($t_1$ is greater than or equal to 0). At $t=0$, you are a passenger at $\bm{\vec{p}} = (x_p, y_p)$, and you walk at speed $v_p$. The bus will pick you up so long as you are on the line segment (the bus doesnt have discrete stops, but will pick you up if it sees you in its path). Can you catch the bus, and if so, over what subset of the line $S$?

 \vspace{0.5cm}\noindent \textbf{Answer: } 
 You can catch the bus if you can reach any point on the route before the bus reaches that same point, or algebraically:
$$A(d) =  t_p(d) - t_b(d) < 0 \mbox{, for some } d \in [0,  \|\bm{\vec{b}_2} - \bm{\vec{b}_1}\|] $$
where 
\begin{itemize}
\item $d$ is the scalar distance the bus has travelled from $\bm{\vec{b}_1}$ towards $\bm{\vec{b}_2}$,
\item $A(d)$ is the amount of time between when the bus arrives and the passenger arrives (if it's negative, the passenger arrives first)
\item $t_p(d)$ is the time at which the \textbf{passenger} arrives at the point along the segment $S$, distance $d$ from the $\bm{\vec{b}_1}$
\item $t_b(d)$ is the time at which the \textbf{bus} arrives at the point along the segment $S$, distance $d$ from the $\bm{\vec{b}_1}$
\end{itemize}

The time at which the bus arrives is just $t_b(d) = t_1 + \frac{d}{v_b}$

The time for the passenger to arrive, $t_p(d)$ is a bit harder. 
For convenience, lets say  $\bm{\vec{\beta}}$ is the unit vector pointing from $\bm{\vec{b}_1}$ to $\bm{\vec{b}_2}$ (that is,  $\bm{\vec{\beta}}$ = $\frac{\bm{\vec{b}_2} - \bm{\vec{b}_1}}{\|\bm{\vec{b}_2} - \bm{\vec{b}_1}\|}$).
$$ t_p(d) = \frac{ \| (\bm{\vec{b}_1} + d\bm{\vec{\beta}}) - \bm{\vec{p}} \|} {v_p} $$

Therefore, we can catch the bus iff
$$ \frac{ \| \bm{\vec{b}_1} + d\bm{\vec{\beta}} - \bm{\vec{p}} \|} {v_p} - \left(t_1 + \frac{d}{v_b}\right) < 0 \mbox{, for some } d \in [0,  \|\bm{\vec{b}_2} - \bm{\vec{b}_1}\|] $$

say $\alpha = \bm{\vec{b}_1} - \bm{\vec{p}}$
  
$$ \frac{ \sqrt{ ||\bm{\vec{\alpha}}||^2 + d^2||\bm{\vec{\beta}}||^2 + 2d(\bm{\vec{\alpha}} \cdot \bm{\vec{\beta}}) }} {v_p} - \left(t_1 + \frac{d}{v_b}\right) < 0 \mbox{, for some } d \in [0,  \|\bm{\vec{b}_2} - \bm{\vec{b}_1}\|] $$

Square some shit:

$$ \frac{ ||\bm{\vec{\alpha}}||^2 + d^2||\bm{\vec{\beta}}||^2 + 2d(\bm{\vec{\alpha}} \cdot \bm{\vec{\beta}}) } {v_p^2} - \left(t_1 + \frac{d}{v_b}\right)^2 < 0 \mbox{, for some } d \in [0,  \|\bm{\vec{b}_2} - \bm{\vec{b}_1}\|] $$

Expand:

$$ \frac{ ||\bm{\vec{\alpha}}||^2 + d^2||\bm{\vec{\beta}}||^2 + 2d(\bm{\vec{\alpha}} \cdot \bm{\vec{\beta}}) } {v_p^2} - t_1^2 - \frac{d^2}{v_b^2} - 2 \frac{dt_1}{v_b} < 0 \mbox{, for some } d \in [0,  \|\bm{\vec{b}_2} - \bm{\vec{b}_1}\|] $$

Group some like terms:

$$ \left(\frac{||\bm{\vec{\beta}}||^2}{v_p^2}  - \frac{1}{v_b^2}\right) d^2 + 
\left(\frac{2(\bm{\vec{\alpha}} \cdot \bm{\vec{\beta}}) } {v_p^2} - 2 \frac{t_1}{v_b}\right) d +
\left(\frac{ ||\bm{\vec{\alpha}}||^2}{v_p^2} - t_1^2 \right)
  < 0 \mbox{, for some } d \in [0,  \|\bm{\vec{b}_2} - \bm{\vec{b}_1}\|] $$

Now we... *checks notes from 7th grade algebra about quadratics*... go about solving this quadratic. 

\section*{Problem 2: }
You're on a bus that will traverse a line segment $S$ from $\bm{\vec{b}_1} = (x_1, y_1)$ to  $\bm{\vec{b}_2} = (x_2, y_2)$ at speed $v_b$. You can get off at any time and walk the rest of the way to your destination $\bm{\vec{d}}$ (likely off of the line segment $S$) at speed $v_p$ (assume $v_p < v_b$). At what time should you get off in order to get to your destination as fast as possible?

\vspace{0.5cm} \noindent \textbf{Answer:}
You should get off when the bus' velocity component in the direction of your destination drops below your walking speed. As before, for convenience, lets say $\bm{\vec{\beta}} = \frac{\bm{\vec{b}_2} - \bm{\vec{b}_1}}{\|\bm{\vec{b}_2} - \bm{\vec{b}_1}\|}$ is a unit vector pointed along the bus's path. Algebraically, this is 

$$\frac{\bm{\vec{v}} \cdot \bm{\vec{u}(t)}}{\|\bm{\vec{u}(t)}\|}  < v_p$$

where $\bm{\vec{u}(t)} = \bm{\vec{d}} - (\bm{\vec{b}_1} + v_b t\bm{\vec{\beta}})$ and $\bm{\vec{v}} =  v_b\bm{\vec{\beta}}$

Oh boy. Let's do some algebra and plug things in. I'm also going to solve for when these become equal. 


$$\frac{  v_b\bm{\vec{\beta}}  \cdot    (\bm{\vec{d}} - (\bm{\vec{b}_1} + v_b t\bm{\vec{\beta}}))  }{\|  \bm{\vec{d}} - (\bm{\vec{b}_1} + v_b t\bm{\vec{\beta}})  \|}  = v_p$$

Distribute the minus signs...

$$\frac{  v_b\bm{\vec{\beta}}  \cdot    (\bm{\vec{d}} - \bm{\vec{b}_1} - v_b t\bm{\vec{\beta}})  }{\|  \bm{\vec{d}} - \bm{\vec{b}_1} - v_b t\bm{\vec{\beta}}  \|}  = v_p$$

Say $\bm{\vec{\alpha}} = \bm{\vec{d}} - \bm{\vec{b}_1}$

$$\frac{  v_b\bm{\vec{\beta}}  \cdot    (\bm{\vec{\alpha}} - v_b t\bm{\vec{\beta}})  }{\|  \bm{\vec{\alpha}} - v_b t\bm{\vec{\beta}}  \|}  = v_p$$

Distribute the dot product.
$$\frac{  v_b\bm{\vec{\beta}}  \cdot    \bm{\vec{\alpha}} - v_b^2 t  }{\|  \bm{\vec{\alpha}} - v_b t\bm{\vec{\beta}}  \|}  = v_p$$

Move the denominator to the other side. 
$$ v_b\bm{\vec{\beta}}  \cdot    \bm{\vec{\alpha}} - v_b^2 t  = v_p \|  \bm{\vec{\alpha}} - v_b t\bm{\vec{\beta}}  \|$$

Square both sides:
$$ (v_b\bm{\vec{\beta}}  \cdot    \bm{\vec{\alpha}} - v_b^2 t)^2  = v_p^2  (\bm{\vec{\alpha}} - v_b t\bm{\vec{\beta}}) \cdot (\bm{\vec{\alpha}} - v_b t\bm{\vec{\beta}}) $$


\section*{Problem 3}
There is some set of buses $B$, where each bus follows some known periodic route consisting of line segments at speed $v_b$. You can also walk in arbitrary directions at speed $v_p$ (which is less than $v_b$). How do we calculate the optimal way to get from point $\bm{\vec{p}_1}$ to $\bm{\vec{p}_2}$?

 \vspace{0.5cm}\noindent \textbf{Partial Answer:}
(for the case where the perons will take 0 or 1 bus, but will not transfer between  buses.)

First, calculate the time it would take for a person to walk there, $t_w$. This is an upper bound on how quickly the person can get there. 

For each bus $b$:
	1. calculate the time that the bus $b$ will begin traversing each line segment of its route, up to $t_w$.
	2. use the solution to problem 1 (above) to figure out which line segments the passenger can walk to.
	3. 


\subsection*{implementation notes}
While it's fine for the very idealized case, the passenger should probably *not* walk to the first point on the route that they can reach, because by definition, they will arrive at THE EXACT SAME TIME AS THE BUS. Instead, they probably should either:
1. walk to the nearest point
2. walk to the point that gives them the largest time margin (their arrival time, minus the bus' arrival time).
I don't know which of these it should be though.

\end{document}
